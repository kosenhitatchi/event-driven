$\vert$ Read the \href{http://robotology.github.io/event-driven/doxygen/doc/html/index.html}{\tt Documentation} $\vert$ Download the \href{https://github.com/robotology/event-driven}{\tt Code} $\vert$

\section*{The event-\/driven Y\+A\+RP Project}

\href{https://youtu.be/xS-7xYRYSLc}{\tt }  Click to watch the \href{https://youtu.be/xS-7xYRYSLc}{\tt video}!

Libraries that handle neuromorphic sensors, such as the dynamic vision sensor, installed on the i\+Cub can be found here, along with algorithms to process the event-\/based data. Examples include, optical flow, corner detection and ball detection. Demo applications for the i\+Cub robot, and tutorials for running them, include saccading and attention, gaze following a ball, and vergence control.

\subsection*{Libraries}

Event-\/driven libraries provide basic functionality for handling events in a Y\+A\+RP environment. The library has definitions for\+:
\begin{DoxyItemize}
\item codecs to encode/decode events to be compatable with address event representation (A\+ER) formats.
\item Sending packets of events in {\ttfamily \hyperlink{classev_1_1vBottle}{ev\+::v\+Bottle}} that is compatible with yarpdatadumper and yarpdataplayer.
\item asynchronous reading ({\ttfamily \hyperlink{classev_1_1queueAllocator}{ev\+::queue\+Allocator}}) and writing ({\ttfamily \hyperlink{classev_1_1collectorPort}{ev\+::collector\+Port}}) ports that ensure data is never dropped and giving access to delay information.
\item filters for removing salt and pepper noise.
\item event containers for organising the event-\/stream into temporal windows, fixed-\/size windows, surfaces, and regions-\/of-\/interest.
\item helper functions to handle event timestamp wrapping and to convert between timestamps and seconds.
\end{DoxyItemize}

\subsection*{Modules}


\begin{DoxyItemize}
\item {\bfseries Optical Flow} -- an estimate of object velocity in the visual plane is given by the rate at which the spatial location of events change over time. Such a signal manifests as a manifold in the spatio-\/temporal event space. Local velocity can be extracted by fitting planes to these manifolds. The {\ttfamily ev\+::v\+Flow} module converts the ED camera output {\ttfamily \hyperlink{classev_1_1AddressEvent}{ev\+::\+Address\+Event}} to {\ttfamily \hyperlink{classev_1_1FlowEvent}{ev\+::\+Flow\+Event}}.
\item {\bfseries Cluster Tracking} -- The movement of an object across the visual field of an ED camera produces a detailed, unbroken trace of events. Local clusters of events can be tracked by updating a tracker position as new events are observed that belong to the same trace. The spatial distribution of the events can be estimated with a Gaussian distribution. The cluster centre and distribution statistics is output from the {\ttfamily ev\+::v\+Cluster} module as a {\ttfamily \hyperlink{classev_1_1GaussianAE}{ev\+::\+Gaussian\+AE}} event.
\item {\bfseries Corner Detection} -- using an event-\/driven Harris algorithm, the full event stream is filtered to contain only the events falling on the corners of objects or structure in the scene. Corner events are useful to avoid the aperture problem and to reduce the data stream to informative events for further processing. Compared to a traditional camera, the ED corner algorithm requires less processing. {\ttfamily ev\+::v\+Corner} converts {\ttfamily \hyperlink{classev_1_1AddressEvent}{ev\+::\+Address\+Event}} to {\ttfamily \hyperlink{classev_1_1LabelledAE}{ev\+::\+Labelled\+AE}}.
\item {\bfseries Circle Detection} -- detection of circular shapes in the event stream can be performed using an ED Hough transform. As the camera moves on a robot, many background events clutter the detection algorithm. The {\ttfamily ev\+::v\+Circle} module reduces the false positive detections by using optical flow information to provide a more accurate understanding of only the most up-\/to-\/date spatial structure. {\ttfamily ev\+::v\+Circle} accepts {\ttfamily \hyperlink{classev_1_1AddressEvent}{ev\+::\+Address\+Event}} and {\ttfamily \hyperlink{classev_1_1FlowEvent}{ev\+::\+Flow\+Event}} and outputs {\ttfamily \hyperlink{classev_1_1GaussianAE}{ev\+::\+Gaussian\+AE}}.
\item {\bfseries Particle filtering} -- probabilistic filtering is used to provide a robust tracking over time. The particle filter is robust to variations in speed of the target by also sampling within the temporal dimension. A observation likelihood function that responds to a circular shape was developed to instigate a comparison with the Hough transform. The tracking position is output as {\ttfamily \hyperlink{classev_1_1GaussianAE}{ev\+::\+Gaussian\+AE}}. Future work involves adapting the filter to respond to different target shapes, and templates learned from data.
\end{DoxyItemize}

\subsection*{Applications for the i\+Cub Humanoid Robot}

Tutorials for these applications can be found \href{http://robotology.github.io/event-driven/doxygen/doc/html/pages.html}{\tt here}


\begin{DoxyItemize}
\item viewing the event-\/stream in 3D spatio-\/temporal space.
\item calibrating the event-\/camera with a static fiducial.
\item performing saccading and attention.
\item ball detection and i\+Cub gaze following.
\item performing automatic stereo vergence.
\end{DoxyItemize}

Datasets for use in running some of the tutorials off-\/line can be found on the same page.

\subsection*{How to Install\+:}

Comprehensive instructions available if you are a first-\/time user of Y\+A\+RP \href{http://robotology.github.io/event-driven/doxygen/doc/html/pages.html}{\tt here}.

Quick instructions\+:


\begin{DoxyEnumerate}
\item Install \href{https://github.com/robotology/yarp}{\tt Y\+A\+RP} and \href{https://github.com/robotology/icub-contrib-common}{\tt icub-\/contrib-\/common} following these \href{http://wiki.icub.org/wiki/Linux:Installation_from_sources}{\tt instructions}.
\item git clone \href{https://github.com/robotology/event-driven.git}{\tt https\+://github.\+com/robotology/event-\/driven.\+git}
\item cd event-\/driven
\item mkdir build \&\& cd build
\item ccmake ..
\item ensure the install path is as configured in icub-\/contrib-\/common
\item turn on desired modules and applications (e.\+g. processing)
\item configure (c) and generate (g)
\item make install
\end{DoxyEnumerate}

\subsection*{References}

Glover, A., and Bartolozzi C. (2016) {\itshape Event-\/driven ball detection and gaze fixation in clutter}. In I\+E\+E\+E/\+R\+SJ International Conference on Intelligent Robots and Systems (I\+R\+OS), October 2016, Daejeon, Korea. {\bfseries Finalist for Robo\+Cup Best Paper Award}

Vasco V., Glover A., and Bartolozzi C. (2016) {\itshape Fast event-\/based harris corner detection exploiting the advantages of event-\/driven cameras}. In I\+E\+E\+E/\+R\+SJ International Conference on Intelligent Robots and Systems (I\+R\+OS), October 2016, Daejeon, Korea.

V. Vasco, A. Glover, Y. Tirupachuri, F. Solari, M. Chessa, and Bartolozzi C. {\itshape Vergence control with a neuromorphic i\+Cub. In I\+E\+E\+E-\/\+R\+AS International Conference on Humanoid Robots (Humanoids)}, November 2016, Mexico. 